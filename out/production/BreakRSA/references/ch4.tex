\input pictex
\input /Users/Shared/TeX/defs
\input /Users/Shared/TeX/ruled

% manually set starting page number and title
%   and numbering on EE's and Projects
\startDoc{CS 4050 Spring 2019}{52}{Chapter 4:  RSA}

% first Exercise is 12
% first Project is 4

{\bigboldfont  Some Basic Number Theory Algorithms and RSA Encryption}
\medskip

In this chapter we'll learn several algorithms that work with integers, and see how these
can be used to do public key encryption by the RSA approach.
This work will use the divide and conquer and the slightly different reduce and conquer
algorithm design techniques for
 its two crucial algorithms, use $Z_n$ concepts, and will multiply large integers
 suggesting that Karatsuba's algorithm would be useful,
so this is a reasonable
time to study it.
\medskip

\Indent{.5}
    This chapter presents an almost complete technical description of RSA encryption.
    Many people present themselves as experts in ``cybersecurity,'' but have no idea how
    something like RSA actually works, because they haven't been forced to endure a chain
    of mathematics courses up to MTH 3170, and they haven't taken a course in advanced
    algorithms.  Of course, our focus---and specifically what you are actually expected to
    fully understand---will be several algorithms.   You will want to understand enough of the
    mathematics to understand the entire process, but some of that material----which will
    be clearly identified as ``optional''---is presented just in case you care.
\border
\Outdent

{\bigboldfont Overview of RSA Encryption}
\medskip

In RSA encryption, the central entity that is being sent encrypted 
messages (``headquarters'') publishes
large, specially chosen integers $n$ and $e$.  To send a message---which is itself a
large integer $a$ that is smaller than $n$---the sender computes, in $Z_n$,
the quantity $c = a^e$, and
sends $c$ as the encrypted message.  At headquarters, the quantity $c^d$ is computed,
in $Z_n$,
and this turns out to be $a$.  The integer $d$ is very specially chosen relative to $n$ and $e$,
and is kept secret from everyone outside headquarters.
\medskip

The value $n$ is taken as the product of two large prime numbers, and the entire security
of the method rests on the idea that no algorithm is known for computing the factors
of $n$ in a reasonable amount of time.
\medskip

To understand how to implement this process efficiently, and how to get the special
number $d$ (and, for that matter, $n$ and $e$), and to prove that it all works, we will
need to cover several fundamental algorithms and appreciate some
theorems from number theory.
\border

{\bf Introductory Case Study}
\medskip

Let's begin by doing RSA naively---without the two main algorithms that we will shortly study.
Suppose at headquarters we pick $p=7$ and $q=13$ and form $n = pq = 91$.
Then we need to pick $e$, which can be any number that has no factors in common with
$$
\phi = (p-1)(q-1) = 6\cdot 12 = 72.
$$
So, say we pick $e=11$ (picking a prime number for $e$ is always
safe, as long as it is big enough to
not itself be a factor of $\phi$).
\medskip

We publish, say on our web page, $n=91$ and $e=11$, along with clear instructions telling
people how to send us a message $a$:  simply compute $a^e$ in $Z_n$ (actually, we would
give people a tool for doing this, since they, being cybersecurity experts and not algorithm and
mathematics experts, are unlikely to be happy doing this for themselves).
\medskip

Say, for this case study, that someone wants to send the message $2$.
They need to compute $2^{11}$ in $Z_{91}$, which is easy---using a calculator, they
compute
$$
2^{11} = 2048,
$$
and then they divide $2048$ by $91$, obtaining $22.505494505494505$.
After some thought, they figure out that what they want is the remainder, which is the
decimal part, so they subtract $22$, obtaining $.505494505494505$, and then multiply by
$91$, obtaining $46$.  We label this number, which is the encrypted message, by $c$.
\medskip

They send $c=46$ to headquarters, where we need to decrypt it.  To do this, we need, as will
be explained shortly, a magical number $d$ such that $de = 1$ in $Z_\phi$.
\medskip

We will find such a number in a naive, horribly inefficient way.  Since
$\phi = 72$, and $e=11$, we want a multiple of $11$ that is one more than a multiple of $72$.
We can simply generate integers that are one more than a multiple of $72$, namely 
$73$, $145$, $217$, and so on, checking each to see whether it is divisible by $11$.
Pretty quickly we find $649 = 9 \cdot 72 + 1 = 59 \cdot 11$.
We see that $59 \cdot 11 $ is equal to $1$ in $Z_{72}$, because multiples of $72$ are effectively $0$.
\medskip

To decrypt the message $c=46$ at headquarters, we now
want to compute $c^d = 46^{59}$ in 
$Z_{91}$.  Unfortunately, this number---$46^{59}$---is too large and can't be computed to full
accuracy on our calculator.  We could use Mathematica or something, but we'll instead do a
terrible algorithm that is still kind of clever:  we simply start computing powers of $46$,
noting that to get each successive row of our table, we multiply the row above by $46$:
\medskip
\ruledtable
$k$ | $46^k$ in $Z_{91}$ \CR
1 | 46 \cr
2 | $46^2 = 2116 = 23$ \cr
3 | $23\cdot 46 = 1058 = 57$ \cr
4 | 74 \cr
5 | 37 \cr
6 | 64 \cr
7 | 32 \cr
8 | 16 \cr
9 | 8 \cr
10 | 4 \cr
11 | 2 \cr
12 | 1
\endruledtable
\medskip

We stop there because now, knowing that $46^{12} = 1$ in $Z_{91}$,
we can cleverly observe that
$$
46^{59} = 46^{48+11} = 46^{12 \cdot 4} \cdot 46^{11}
= \left(46^{12}\right)^4 \cdot 46^{11}
= (1)^4 \cdot 46^{11}
= 2,
$$
which is the original message!
\medskip

This case study has identified the need for several algorithms.
\medskip

\Indent{.5}
First, when computing $a^e$ in $Z_n$, and when computing $c^d$ in $Z_n$, we
need an efficient algorithm for performing this {\it modular exponentiation\/} calculation.
Note that the way we computed $46^{59}$ in $Z_{91}$ would be hopelessly time-consuming
with large integers (even with small integers, it could have required us to keep going in our
table up to $59$ to get the desired answer, as far as we know---we lucked out by finding a
much smaller power---$12$---for which we got $1$ as the answer).
\medskip

Second, we need an efficient algorithm for finding the magical $d$---obviously the brute force
approach of just trying multiples of $\phi$ until we find one that is $1$ below a multiple of $e$
is hopeless.
\medskip

In the rest of the chapter we will develop these two algorithms, and prove that they produce the
desired answers even when $n$, $e$, $\phi$, $c$, and $d$ are huge---like thousands of digits long.
\medskip
\Outdent

Later we will prove, for anyone interested, that the method works.
\medskip

So, this case study has given us a map for most of the rest of the chapter, namely
developing an efficient algorithm to do modular exponentiation, developing an efficient
algorithm to find the inverse of $e$ in $Z_{\phi}$, and proving some number theory facts.
\medskip

The case study has not foreshadowed one other issue that we will discuss:  how does
headquarters come up with large prime numbers to use for $p$ and $q$?
\border

{\bigboldfont Modular Exponentiation}
\medskip

Our first new problem in this chapter is {\it modular exponentiation\/}, whose efficient
solution is crucial for RSA encryption:  given positive
integers $a$ and $n$, where $a<n$, and any positive integer $k$,
compute $a^k$ in $Z_n$.
\medskip

% obvious/brute force algorithm:
The fairly obvious algorithm for solving this problem would be:
\medskip
\verbatim
public int modExp( int a, int k, int n )
{
  int r = 1;  
  for( int j=1; j<=k; j++ )
    r = (r * a) % n; 
  return r;
}
|endverbatim
\medskip

Unfortunately, this algorithm has efficiency in $\Theta(k)$, and
$k$ can be very large, so this is not acceptable.
\border

{\bf A More Efficient Algorithm for Modular Exponentiation}
\medskip

Suppose for simplicity that $k=2^m$.  Then to compute $a^k = a^{2^m}$ in $Z_n$, we can use
the divide-and-conquer approach.
\medskip

We realize that if we compute $a^{k/2} = a^{2^{m-1}}$ in $Z_n$, then we can simply square that
value to compute the desired value, since $a^{k/2} a^{k/2} = a^k$.  
If we do this recursively all the way down, we have
an algorithm that takes only $m$ operations to compute the value, rather than $2^m$.
\medskip

In this case, it is easier to implement the recursion
iteratively---we compute $a^1$, then square that to get $a^2$,
square that to get $a^4$, and so on, until we reach $a^{2^m} = a^k$.
\medskip

And, of course, at each step we reduce the result mod $n$ so that the
numbers never get larger than $n^2$.
\medskip

Once we have all the values of $a$ raised to all the relevant powers of $2$, like
$a^1$, $a^2$, $a^4$, $\ldots$, we can then express $k$ as a sum of powers of $2$---basically
just expressing it in binary notation---and then multiply together the terms that are needed
to produce $a^k$.
\medskip

Of course, if $k$ is not a power of $2$, we need to adjust the algorithm somewhat, but
it will still be very efficient.
\medskip

It is obvious that this algorithm requires $m$ steps, and since $m = \logtwo k \leq \logtwo n$,
the algorithm takes $O( \logtwo n)$ steps.
And, each step involves multiplying together two integers of size at most $n$
and reducing the result mod $n$, which is manageable.
\medskip

Note that for this problem (doing RSA encryption), we are not interested in the
efficiency analysis to compare to alternative algorithms as $n$ gets arbitrarily
large.  Instead, we want to make sure that for a fairly large $n$ (one for which it
is hopeless to factor $n$), all the work can be done in a reasonable amount of time.
For example, if $n$ is around 1024 bits in size, then the modular exponentiation algorithm
we are sketching out takes 10 steps, and each step takes less than a million single bit
multiplications and a few thousand  single bit additions (using Karatsuba, say),
so the total work to compute $a^k $ in $Z_n$ can be done in a few seconds.
\border

{\bf Example of Efficient Modular Exponentiation Algorithm by Hand}
\medskip

First let's compute $a^e$ in $Z_n$ where $a=73$, $e=29$, and $n=151$.
The main idea is that since $ 29 = 16 + 8 + 4 + 1$,
$$
73^{29} = 73^{16} \cdot 73^8 \cdot 73^4 \cdot 73^1,
$$
and we note that $a^{2m} = (a^m)^2$, so we can compute all the
values of $a$ raised to a power of 2 by repeated squaring,
so in $Z_{151}$ we compute 
\medskip

\ruledtable
$73^1 $ & & &$= 73$ \cr
$73^2$ & & $= 5329 $&$= 44$ \cr
$73^4 $& $= 44^2 $&$= 1936$&$ = 124$ \cr
$73^8 $ &$ = 124^2$&$ = 15376$&$ = 125 $\cr
$73^{16}$ &$ = 125^2$&$ = 15625$&$ = 72$
\endruledtable
\medskip

Now that we have all the values of $a$ raised to the necessary powers of $2$,
we can multiply them together to obtain the answer, which is
$$
73^{16} \cdot 73^8 \cdot 73^4 \cdot 73^1 = 72 \cdot 125 \cdot 124 \cdot 73 
=  27.
$$
\border

{\bf The Final Slick Modular Exponentiation Algorithm}
\medskip

All we have to do now is remember how to convert a given decimal integer into
binary and integrate that into our successive squares algorithm.
\medskip

The idea of converting a given integer $k$ into binary is to note that
if $k$ is even, its last (or least significant or rightmost) bit is 0, and
if $k$ is odd, that bit is 1.
Then we divide $k$ by 2 and use the same idea to get the rightmost bit
of $k/2$, which is the $2$'s bit of $k$, and so on.
\medskip

So, our final slick algorithm is demonstrated here for the same problem as
was done previously.
We can make a chart where the first column contains
the successive squares of $a$, the second column holds the repeated halvings of
$e$, using integer arithmetic, and the third column contains the cumulative products of the 
items in the first column where the second column contains an odd number.
\medskip

Doing this gives us:
\medskip

\verbatim
 73   29    73
 44   14
124    7   143
125    3    57
 72    1    27
|endverbatim
\border

{\bigboldfont Exercise 12} (efficient modular exponentiation by hand)
\medskip

Perform the efficient modular exponentiation algorithm using a calculator for
$a=29613$, $e=8075$, working in $Z_{31861}$.
\medskip

The folder {\tt Code/RSA} at the course web site contains a class {\tt ModExp} that
does this algorithm, but only use it to check your work---remember that
you will be asked to
do this computation on the test with just a calculator.
\border

\vfil\eject

{\bigboldfont The Extended Euclidean Algorithm}
\medskip

We now need to study the Euclidean algorithm, which may be one of the oldest published
algorithms.  The Euclidean algorithm provides an efficient solution to the problem ``find the
greatest common divisor of two given positive integers.''
\medskip

The Euclidean algorithm will be the tool that lets us efficiently obtain the magical
decrypting number $d$ when given $n=pq$ and $e$.
\medskip

Recall that a positive integer $d$ divides, or is a divisor of, a positive integer $a$ if and only if
there is some integer $q$ such that $a = q d$.  The greatest common divisor
of $a$ and $b$, written $\hbox{gcd}(a,b)$, is the largest integer that is a divisor of both $a$
and $b$.
\medskip

In elementary school you might have learned this concept when working with fractions---to
reduce a fraction, you can find the greatest common divisor of the top and bottom and then
divide both by that number.  More likely you learned to completely factor the top and bottom
into prime factors and then cancel matching factors.  That algorithm was ridiculously
inefficient, it turns out.  We want to compute the {\tt gcd} not to find the greatest common
divisor, because we will only
use it with numbers where the {\tt gcd} is 1, but to find additional crucial information.
\medskip

Given two positive integers $a$ and $b$, we perform integer division on them, obtaining
non-negative integers $q$ and $r$, with $r < b$, such that
$$
a = qb + r.
$$
When you learned to do division in elementary school, you were learning an algorithm to
find $q$ and $r$.
\medskip

{\bf Crucial Fact about the GCD:}  With $q$ and $r$ as above,
$$\hbox{gcd}(a,b) = \hbox{gcd}(b,r).$$
\In {\bf Proof:}
We have to prove two directions, showing that any number that divides $a$ and $b$ also divides $b$ and $r$, 
and that any number that divides $b$ and $r$ also divides $a$ and $b$ (this of course proves that the two 
greatest common divisors are the same):
\medskip

Suppose $d$ divides $a$ and $d$ divides $b$.  Then $d$ divides $b$ (duh!), and since
$r = a-qb$, $d$ clearly divides $r$.
\medskip

For the other direction, suppose $d$ divides $b$ and $d$ divides $r$.  Then since $a = qb+r$, $d$ divides $a$,
and of course $d$ divides $b$.
\bigskip
\Out 

\vfil\eject

Based on this fact we can easily write recursive code to compute the {\tt gcd}:
\medskip
\hrule
\smallskip
{\baselineskip .9\baselineskip
\verbatim
public static int gcd( int a, int b )
{
  if( a == 0 )  return b;
  else if( b == 0 )  return a;
  else{
    int q = a/b, r = a%b;
    return gcd( b, r );
  }
}
|endverbatim
\smallskip}
\hrule
\medskip

{\bf Example}  Here is a chart showing the sequence of recursive calls (the algorithm is easily
written using a loop instead of recursion) for the initial call {\tt gcd(1861,757)}:
\medskip

\ruledtable
$a$ | $b$ | $r$ | $q$  \cr
1861 | 757 | 347 | 2 \cr
757 | 347 | 63 | 2  \cr
347 | 63 | 32 | 5\cr
63 | 32 | 31 | 1 \cr
32 | 31 | 1 | 1 \cr
31 | 1 | 0  | 31 \cr
1 | 0 | | 
\endruledtable
\border

{\bf The Extension}
\medskip
You may have studied this algorithm before, but now we want to use it in a much more
powerful way.  It turns out that a simple extension to this algorithm proves the following fact
and gives us what we need for getting the magical decrypting $d$ for RSA encryption.
\medskip

{\bf Fact}  Given any positive integers $a$ and $b$, there exist integers $s$ and $t$
 (one positive, one
negative, actually) such that 
$$
\hbox{gcd}(a,b) = s a + t b.
$$
\In
{\bf Proof:}  This fact actually follows from the earlier proof that $\hbox{gcd}(a,b) = \hbox{gcd}(b,r)$,
and induction on the number of calls to the recursive function.  
\medskip
The base case is when one of the numbers, say $b$, is 0, and the other, $a$, is not.
In that case, the gcd is the non-zero number, so the desired result is obtained by taking
$s=1$ and $t$ equal to any number whatsoever, say $w$, since
$$ \hbox{gcd}(a,0) = a = 1 \cdot a + w \cdot 0.
$$

For the inductive step, we assume that we have numbers $s'$ and $t'$ such that
$$
\hbox{gcd}(b,r) = s' b + t' r.
$$
To find the desired multipliers $s$ and $t$, such that $ \hbox{gcd}(a,b) = sa + tb$,
we simply note that this expression does not involve $r$, and that $\hbox{gcd}(a,b) =
\hbox{gcd}(b,r)$, so we substitute $r = a-qb$ into the above equation and obtain
$$
\hbox{gcd}(a,b) = \hbox{gcd}(b,r) = s' b + t' r = s' b + t'(a-qb) = t' a + (s' - t'q) b.
$$
Thus, if we take $s=t'$ and $t=s'-t'q$, we have the desired result.
\bigskip
\Out

{\bf Extended Euclidean Algorithm Code}
\medskip

Here is code for the extended Euclidean algorithm.  Note that we want each call to
{\tt gcd} to return three values---the gcd, $s$, and $t$---so we use an array.
\medskip
\hrule
\smallskip
\verbatim
public static int[] gcd( int a, int b )
{
  int[] x = new int[3];
  if( a == 0 )
  {
     x[0] = b;  x[1] = 0; x[2] = 1;
     return x;
  } 
  else if( b == 0 )
  {
    x[0] = a;  x[1] = 1;  x[2] = 0; 
    return x;
  }
  else
  {
    int q = a/b, r = a%b;
    int[] xprime = gcd( b, r );
    
    x[0] = xprime[0];  
    x[1] = xprime[2];  x[2] = xprime[1] - q*xprime[2];
    
    return x;
  }
}
|endverbatim
\border

\vfil\eject

{\bf Example}  Here is a chart showing the sequence of recursive calls (the algorithm is easily
written using a loop instead of recursion) for the initial call {\tt gcd(1861,757)}, with
the returning values of $s$ and $t$ also shown:
\medskip

\ruledtable
$a$ | $b$ | $r$ | $q$  | s | t \cr
1861 | 757 | 347 | 2 | 24 | -59 \cr
757 | 347 | 63 | 2 | -11 | 24 \cr
347 | 63 | 32 | 5 | 2 | -11 \cr
63 | 32 | 31 | 1 | -1 | 2\cr
32 | 31 | 1 | 1 |1 | -1\cr
31 | 1 | 0  | 31 | 0 | 1\cr
1 | 0 | |  | 1 | 0  
\endruledtable
\bigskip

{\bigboldfont Exercise 13} (performing extended Euclidean algorithm by hand)
\medskip

Create the chart for the extended Euclidean algorithm on the instance of the
problem ``compute {\tt gcd( 439, 211 )}.''  
\medskip

First work downward, producing $a$, $b$, $r$, and $q$ for each row.
\medskip

Then work back up, filling in the columns for $s$ and $t$.
For the bottom row, which for our RSA work will always have $a=1$ and $b=0$,
use $s=1$ and $t=0$.
Work upward using the
formulas 
$$
s = t'
$$
$$
t = s' - t'q,
$$
where $s'$ and $t'$ are the values in a row, and $s$ and $t$ are the values in the
row above.
\medskip

Each time you compute $s$ and $t$ for a row, be sure to check that for the
values in that row,  $ sa + tb $ is equal to the final GCD (which will always be $1$ for our RSA work).
\medskip

After doing this, do it again, but this time use $s=1$ and $t=1$ in the bottom row.
Note that we can use $s=1$ and $t=$anything on the bottom row and still have
$sa+tb = 1$, since $b=0$ on the bottom row.
Note how the pattern of alternation reverses.  
Figure out why this algorithm always gives alternating signs in the $s$ and $t$ columns.
\bigskip

Note that the class {\tt GCD} in the {\tt RSA} folder implements this algorithm,
using a choice of $s$ and $t$ for the bottom row that produces a positive value for
$t$, as desired for the RSA work (also note that the column labels are different, maintaining
$ g f + d e = 1$ on each row).
\border

\vfil\eject

{\bf Efficiency Analysis for the Euclidean Algorithm}
\medskip

As was the case with modular exponentiation, if we can't do the Euclidean algorithm
quickly enough for  huge numbers $a$ and $b$, this
whole scheme won't be practical.
The following theorem is somewhat reassuring.
\medskip

{\bf Theorem}  Let the Fibonacci sequence be defined by the recurrence relation\hfil\break
$f_{k+2} = f_{k+1} + f_k$, along with the initial conditions $f_1=1$ and $f_2=1$.
If $n > m \ge 1$, and $\hbox{gcd}(n,m)$ causes $k\ge 1$ recursive calls,
then
$n \ge f_{k+2}$ and $m \ge f_{k+1}$.
\medskip
\In
{\bf Proof:}  We use induction on $k$.
\medskip
For the base case, when $k=1$, if $\hbox{gcd}(n,m)$ causes just one call,
we want to prove that $n \ge f_3 $ and $m \ge f_2$.
Since $f_2 = 1$ and $m \ge 1$ by an assumption of this theorem, that part is easy.
Since $n>m$ by assumption, $n > 1$, so $n \ge f_3 = 2$.
\medskip

\def\mod{\hbox{ mod }}

Now, assume that the desired results are true for $k-1$ calls, for $k\ge 2$, and
suppose that $\hbox{gcd}(n,m)$ causes $k$ calls.
The first call simply calls $\hbox{gcd}( m, n \mod m)$, which must cause $k-1$ calls,
so the induction hypothesis applies, and therefore
$m \ge f_{k+1}$ and $ n \mod m \ge f_k$.
The first inequality is one of the things we need to prove, so yay!
For the other, note that $ n = qm + (n \mod m)$ for some integer $q$, 
and since $n > m$ by assumption, we must have $q \ge 1$, so
$$
n = qm + (n\mod m) \ge m + (n \mod m) \ge f_{k+1} + f_k  =  f_{k+2}
$$
as desired.
\medskip
\Out 
\border

This theorem shows that the worst-case for the time required for the
Euclidean algorithm is when two successive Fibonacci numbers are being worked on.
\medskip

We mentioned earlier that one can show that
the Fibonacci sequence has a growth rate in
 $\Theta(1.6^n)$.  
So, the previous theorem show that the  time for the Euclidean algorithm is in $\Theta\left( \log_{1.6} ( n ) \right)$, meaning that
this part of the RSA scheme is quite feasible.
\border

{\bf Building the RSA Scheme}
\medskip

Now we can recall how to create an RSA scheme and see where we stand.
\medskip

First, we come up with two large prime numbers $p$ and $q$  (how we do this
will be covered later) and compute $n=pq$.
We also compute $\phi = (p-1)(q-1)$.
Then we somehow come up with a positive integer $e$ such that $\hbox{gcd}(\phi,e) = 1$,
and at the same time as we check to make sure that $\hbox{gcd}(\phi,e)=1$,
compute $s$ and $d$ such that
$$
1 = s \phi + d e,
$$
with $d > 0$ (we were calling this number $t$ when discussing the Euclidean algorithm).
\medskip

We proved earlier that the extended Euclidean algorithm is efficient enough to use for
the purpose of getting the magical $d$.
\medskip

\In
As seen in Exercise 13,
we have to choose either $s=1$ and $t=0$ or $s=1$ and $t=1$ for the bottom row
to produce $d > 0$.
\medskip
\Out

We have also seen that modular exponentiation is efficient enough to make the
computations of $a^e$ in $Z_n$ and $c^d$ in $Z_n$ feasible.
\border

{\bigboldfont Exercise 14} (the entire RSA process by hand)
\medskip

Work out by hand/calculator and write up a careful example of the entire RSA process, including 
computing $n$ to publish, encrypting $a$, 
finding the secret $d$, and decrypting $c$,
with $p=137$, $q=241$, $e=53$, and $a = 12345$.
You will be expected to do all of this  (including efficient modular exponentiation and
the extended Euclidean algorithm) by hand/calculator during the test, so do the exercise that way,
using just an ordinary
calculator.
\border


\Indent{1}
Note that various classes in the {\tt RSA} folder implement all the parts of
this process.  So, with a little judicious copying and pasting into and out of
email messages, we can communicate securely by email.
\border
\Outdent

{\bigboldfont Project 4} (breaking RSA)
\medskip

Please don't share your answer to this project with anyone else, and don't work in
a group---remember that the
more people who complete a project, the less I will weight it in the course grades.
\medskip

Suppose you intercept the message  
$1027795314451781443748475386257882516$ 
and you suspect it was encrypted using RSA
with 
$n=1029059010426758802790503300595323911$ and $e=2287529$.
Break the encryption and determine the original message.
\medskip

You might find some of the code in the {\tt Code/RSA} folder helpful for this.
You might also find that you need to use {\tt wolfram alpha} to factor $n$ (on my
machine this works in a reasonable time).
Note that I have used the scheme shown in {\tt Encode/Decode} to switch between
integers and strings.
\medskip

Email me your results, including the decrypted plain-text message and details of how you got it.
\border

\vfil\eject
%\bye

{\bigboldfont Fermat's Little Theorem}
\medskip

Earlier we saw that in $Z_n$, computing quantities like $a^k$ leads to some interesting
patterns.  Now we want to prove a famous number theory fact that will let us understand
exactly why the RSA scheme works.  It will also help us to believe the method we will
use for finding large prime numbers.
\medskip

{\bf Theorem}  If $n$ is prime, and $a$ is any positive integer,
then $a^n = a$ in $Z_n$.
\medskip
\Indent{.5}
{\bf Proof:} 
We will prove this by mathematical induction on $a$.
\medskip

The base case, where $a=1$, is trivial:  obviously $1^n = 1$ in $Z_n$.
\medskip

Assume that $a^n = a$ in $Z_n$ for some $a>1$.
We want to show that $(a+1)^n = a+1$ in $Z_n$.
\medskip

\Indent{.5}
Let's do a specific example to motivate the upcoming simple proof.
Use $n=7$ and $a=3$.
If we compute the powers of $3$ in $Z_7$, we get
$3^1=3$, $3^2=2$, $3^3=6$, $3^4=4$, $3^5=5$, $3^6=1$, and $3^7=3$, which shows
that the results holds for $3$.  Let's use that fact to show that it holds for $4$, namely 
that $4^7 = 4$ in $Z_7$.
\medskip

Recall the binomial theorem, which says that 
$$
\scriptstyle
(3+1)^7 = {7 \choose 0} 3^7 + {7 \choose 1} 3^6 + {7 \choose 2} 3^5 + 
{7 \choose 3} 3^4 +
{7 \choose 4} 3^3 +
{7 \choose 5} 3^2 +
{7 \choose 6} 3^1 +
{7 \choose 7} 3^0
$$
We can quickly use Pascal's triangle to obtain these coefficients:
\verbatim
                 1
               1   1
             1   2   1
           1   3   3   1
         1   4   6   4   1
       1   5  10  10   5   1
     1   6  15  20  15   6   1
   1   7  21  35  35  21   7   1
|endverbatim
\medskip

The big observation is that all the binomial coefficients in row $7$
other than the first and last ones
are divisible by $7$, hence equal to $0$ in $Z_7$.
So, we have
$$
(3+1)^7 = 3^7 + 1,
$$
since all the other terms are $0$ in $Z_7$.
But, then since we know the desired property holds---that is, $3^7 = 3$ in $Z_7$,
this means that
$$
(3+1)^7 = 3+1,
$$
or that $4^7 = 4$ in $Z_7$.
\medskip
\Outdent

In our actual situation, we know that
$$
(a+1)^n = \sum_{k=0}^{n} {n \choose k} a^k
= a^n + \left[ \sum_{k=1}^{n-1} {n \choose k} a^k \right] + 1
= a^n + 1
$$
in $Z_n$, because we will prove below that all the quantities $n \choose k$, for $1\leq k \leq n-1$,
are equal to $0$ in $Z_n$.
Then, since the induction hypothesis says that $a^n = a$ in $Z_n$, 
we have our desired result, namely that
$$
(a+1)^n = a +1
$$
in $Z_n$.
\medskip

To finish the proof, we recall that for $1\leq k \leq n-1$,
$$
{n \choose k} = { n! \over k! (n-k)! } = { n (n-1) \cdots (n-k+1) (n-k)! \over k! (n-k)! }
= { n (n-1) \cdots (n-k+1) \over 1 \cdot 2 \cdots k}.
$$
\medskip

\Indent{1}
For example, 
$$
{7 \choose 4 } = {7!  \over 4! 3!   } = { 7 \cdot 6 \cdot 5 \cdot 4 \cdot 3! \over 1 \cdot 2 \cdot 3 \cdot 4 \cdot 3! }
$$
$$ = { 7 \cdot 6 \cdot 5 \cdot 4 \cdot \strikeout{3!} \over 1 \cdot 2 \cdot 3 \cdot 4 \cdot \strikeout{3!} }
= { 7 \cdot \strikeout{6} \cdot 5 \cdot \strikeout{4} \over 1 \cdot \strikeout{2} \cdot \strikeout{3} \cdot \strikeout{4} } = 7 \cdot 5.
$$
\medskip
\Outdent

Then we observe that this fraction must actually end up being an integer after cancelling
all the integers on the bottom with factors in the top.  And, since all the integers on the bottom
are less than $n$, and $n$ is prime, clearly the leading factor $n$ can't be cancelled.
\border
\Outdent

Now we know that 
if $n$ is prime, then
for any positive integer $a$,
$a^n = a$ in $Z_n$, which means $n$ divides $a^n-a = a(a^{n-1} - 1)$.
Then, if $a$ is between $2$ and $n-1$ inclusive (which is how we usually represent
numbers in $Z_n$),
since $n$ is prime and can't have an integer between $2$ and $n-1$
as a factor (it's only factors are $1$ and $n$), that
$n$ divides $a^{n-1} -1 $, or, in other words, $a^{n-1} = 1$ in $Z_n$.
\border

{\bigboldfont Getting Large Primes}
\medskip

In addition to providing facts that will be crucial in showing how and why RSA works,
the preceding theorem gives us some context for our method of producing large prime
numbers.
\medskip

The preceding theorem says that for any $a < n$, if $n$ is prime, then
$ a^{n-1} = 1$ in $Z_n$.
\medskip

The converse turns out to almost be true!  It turns out that if $ 2^{n-1} = 1$ in $Z_n$, then
$n$ is incredibly likely to be a prime number---but not always!
\medskip

\In
   So, this is a reasonable process for obtaining large prime numbers:  take a large integer
   $p$ and
   perform the test (compute $2^{p-1}$ in $Z_p$ and see if you get $1$).  If you do get $1$,
   it is a really safe bet that $p$ is prime.  If you don't get $1$, move on to the next odd number
   and try again.  In a reasonable number of steps like this, you'll find a prime number larger
   than or equal to the place you started.
   \medskip
   
   It has been proven that the distribution of primes is thick enough that this
   process will hit a prime soon enough to be reasonable.
   \medskip
   
   If we are really unlucky, we might pick a so-called pseudo-prime---an integer $p$
   for which the test is passed, but $p$ is not prime.
\border

{\bf Smallest Fermat Psuedo-Prime}
\medskip

If we apply this test to integers $2$, $3$, $4$, and so on, the first one for which the test fails is
$n=341$.
\medskip

We can use our modular exponentiation algorithm to compute $2^{340}$ in $Z_{341}$---here is the
chart produced:
\medskip
\verbatim
              2             340
              4             170
             16              85              16
            256              42
             64              21               1
              4              10
             16               5              16
            256               2
             64               1               1
|endverbatim

which shows that $341$ passes the test.
However, $341$ is not prime, because $341 = 11 \cdot 31$.
\border

\Indent{.5}
   In 2002, Agrawal, Kayal, and Saxena found a polynomial-time algorithm (polynomial
   in the
   number of bits needed to represent $n$) for determining whether a given integer $n$ is prime.
   Before this result (or the fairly recent results leading up to it), this problem might have been
    thought
   to be NP-complete (we will learn what this means toward the end of this course).
   Note that their algorithm does not give factors if $n$ is found to be composite, it only
   says whether it's composite.  We won't attempt to learn about this algorithm, but you
   should be aware of its existence.
\border
\Outdent

The {\tt BigInteger} class method {\tt isProbablePrime} gives a more reliable way to
determine whether a given integer is probably a prime number (probably uses the
Miller-Rabin test, a more accurate but still efficient probabilistic test for primality).
\border

\vfil\eject

{\bigboldfont A Proof that RSA Encryption/Decryption Works}
\medskip

We are now ready to understand RSA encryption/decryption.  Specifically, we want to prove that it
works.
\medskip

Recall that at headquarters large primes $p$ and $q$ are picked (we hope in a way that
outsiders won't be able to reverse-engineer---for example, deciding to use $n$ with
$M$ digits
and then using primes near the square root of $10^M$ would be a bad idea), 
and $n = pq$ and $\phi = (p-1)(q-1)$
are computed.  A number $e$ is picked, such that $ \hbox{gcd}( \phi, e ) = 1$,
and the extended Euclidean algorithm is used to find $s$ and $d$, with $s<0$ and $d > 0$,
such that  $ 1 = s\phi + de$.
\medskip

The values $n$ and $e$ are published.  As far as anyone on Earth knows (or at least
is admitting publicly), there is
no algorithm for using $n$ to determine $\phi$, or $p$ or $q$, in a reasonable amount of
time (assuming $p$ and $q$ are large enough, reasonable is something like ``before the
sun goes super-nova'').
\medskip

Someone out in the field wants to send the message $a$ to headquarters.  They
simply compute $ c = a^e $ in $Z_n$ and send $c$ to headquarters.
\medskip

Back at headquarters, they simply compute $ c^d $ in $Z_n$ and get $a$ back.
\medskip

We now want to show that this works.  In all the following, note that since $s<0$, $-s$ is a
positive integer, so using it as an exponent makes sense without worrying about multiplicative
inverses.
\medskip

Since $c=a^e$ in $Z_n$, there is some integer $g$ such that
$ c = a^e + gn$ in ordinary arithmetic.
So,
using ordinary integer arithmetic (this is done for the proof, but not in the algorithm),
since $c = a^e + gn$,
$$
c^d = (a^e + gn)^d = (a^e)^d + hn$$
for an integer $h$ obtained by using the binomial theorem (in $(a+b)^n$, all terms except the
$a^n$ term have $b$ as a factor).
So,
$$
c^d =
a ^ {ed} + hn = a ^ { 1 - s \phi } +hn  = a^1 a ^ {-s\phi} + hn = a a^{-s\phi} +hn.
$$
Now, in $Z_p$, 
$hn = 0$ (since $p$ is a factor of $n$), and so
$$
a a^ {-s \phi } + hn = a a ^ {-s (p-1)(q-1)} = a (a^{p-1})^{-s(q-1)} = a 1^{-s(q-1)} = a,
$$
by Fermat's little theorem (which says that $ a^{p-1} = 1$ in $Z_p$).
So, in $Z_p$, $ c^d = a$.
\medskip

Similarly, in $Z_q$, $hn=0$, so
$$
a a^ {-s \phi } +hn  = a a ^ {-s (p-1)(q-1)} = a(a^{q-1})^{-s(p-1)} =  a 1^{-s(p-1)} = a,
$$
so in $Z_q$, $c^d = a$.
\medskip

So, we know that $ c^d = a $ in $Z_p$, and $c^d = a$ in $Z_q$.  Now we need to show that
$c^d = a$ in $Z_n$.
\medskip

Since $c^d = a$ in $Z_p$, there is some integer $\alpha$ such that $c^d-a = \alpha p$
in ordinary arithmetic.
Similarly, since $c^d=a$ in $Z_q$, there is some integer $\beta$ such that $c^d-a = \beta q$
in ordinary arithmetic.
So, in ordinary arithmetic $\alpha p = \beta q$, which shows that
$p$ is a factor of $\beta q$.
Since $p$ and $q$ are primes, this can only happen if $p$ is a factor of $\beta$, which means there
is some integer $\gamma$ such that $\beta = \gamma p$.
So,
$$
c^d-a = \beta q = \gamma p q = \gamma n
$$
in ordinary arithmetic,
so $c^d = a$ in $Z_n$ as desired.
\border

 \vfil\eject
 \bye
